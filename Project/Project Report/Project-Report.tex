% Options for packages loaded elsewhere
\PassOptionsToPackage{unicode}{hyperref}
\PassOptionsToPackage{hyphens}{url}
%
\documentclass[
]{article}
\usepackage{amsmath,amssymb}
\usepackage{lmodern}
\usepackage{iftex}
\ifPDFTeX
  \usepackage[T1]{fontenc}
  \usepackage[utf8]{inputenc}
  \usepackage{textcomp} % provide euro and other symbols
\else % if luatex or xetex
  \usepackage{unicode-math}
  \defaultfontfeatures{Scale=MatchLowercase}
  \defaultfontfeatures[\rmfamily]{Ligatures=TeX,Scale=1}
\fi
% Use upquote if available, for straight quotes in verbatim environments
\IfFileExists{upquote.sty}{\usepackage{upquote}}{}
\IfFileExists{microtype.sty}{% use microtype if available
  \usepackage[]{microtype}
  \UseMicrotypeSet[protrusion]{basicmath} % disable protrusion for tt fonts
}{}
\makeatletter
\@ifundefined{KOMAClassName}{% if non-KOMA class
  \IfFileExists{parskip.sty}{%
    \usepackage{parskip}
  }{% else
    \setlength{\parindent}{0pt}
    \setlength{\parskip}{6pt plus 2pt minus 1pt}}
}{% if KOMA class
  \KOMAoptions{parskip=half}}
\makeatother
\usepackage{xcolor}
\usepackage[margin=1in]{geometry}
\usepackage{graphicx}
\makeatletter
\def\maxwidth{\ifdim\Gin@nat@width>\linewidth\linewidth\else\Gin@nat@width\fi}
\def\maxheight{\ifdim\Gin@nat@height>\textheight\textheight\else\Gin@nat@height\fi}
\makeatother
% Scale images if necessary, so that they will not overflow the page
% margins by default, and it is still possible to overwrite the defaults
% using explicit options in \includegraphics[width, height, ...]{}
\setkeys{Gin}{width=\maxwidth,height=\maxheight,keepaspectratio}
% Set default figure placement to htbp
\makeatletter
\def\fps@figure{htbp}
\makeatother
\setlength{\emergencystretch}{3em} % prevent overfull lines
\providecommand{\tightlist}{%
  \setlength{\itemsep}{0pt}\setlength{\parskip}{0pt}}
\setcounter{secnumdepth}{-\maxdimen} % remove section numbering
\ifLuaTeX
  \usepackage{selnolig}  % disable illegal ligatures
\fi
\IfFileExists{bookmark.sty}{\usepackage{bookmark}}{\usepackage{hyperref}}
\IfFileExists{xurl.sty}{\usepackage{xurl}}{} % add URL line breaks if available
\urlstyle{same} % disable monospaced font for URLs
\hypersetup{
  pdftitle={Project Report},
  pdfauthor={Trinath Sai Subhash Reddy Pittala, Uma Maheswara R Meleti, Hemanth Vasireddy},
  hidelinks,
  pdfcreator={LaTeX via pandoc}}

\title{Project Report}
\author{Trinath Sai Subhash Reddy Pittala, Uma Maheswara R Meleti,
Hemanth Vasireddy}
\date{2023-04-04}

\begin{document}
\maketitle

\hypertarget{introduction}{%
\section{Introduction}\label{introduction}}

\hypertarget{airbnb-price-determinants-in-europe}{%
\subsection{Airbnb Price Determinants in
Europe}\label{airbnb-price-determinants-in-europe}}

We want to work on Airbnb's dataset from kaggle.com. It provides
information about hotel rooms in Europe.

Each major city has its dataset for weekends and weekdays Variables
included in the dataset: Host ID (Id) The total price of listing
(realSum) Room type: private, shared, entire home, apt (room\_type)
Whether or not a room is shared (room\_shared) Max number of people
allowed in property (person\_capacity) Whether or not the host is
superhost (host\_is\_superhost) Whether or not it is multiple rooms
(multi) Whether for business or family use (biz) Distance from the city
center (dist) Distance from nearest metro (metro\_dist) Latitude and
longitude (lat long) Guest satisfaction (guest\_satisfaction\_overall)
Cleanliness (cleanliness\_rating) The total quantity of bedrooms
available among all properties for a single host (bedrooms)

Questions we can answer with the dataset: Price Forecasting: use
pricing, room type, and amenities to predict potential rental prices in
the future. Hotspots: use listing location in relation to business and
tourism centers and correlate this with pricing to determine where
Airbnb rentals would be most profitable Customer Sentiment Analysis:
analyze customer comments and satisfaction ratings to evaluate listing
on overall customer experience and use it to optimize hosts' services to
improve user satisfaction ratings.

How can this information be used: Data can help travelers find
accommodation that meets their needs without exceeding budget. Can help
hosts set competitive pricing and optimize listings to get more
bookings. Help investors evaluate the value of investing in real estate
in different European cities based on pricing trends.

\hypertarget{variable-description}{%
\subsection{Variable Description}\label{variable-description}}

realSum: The total price of the Airbnb listing

roomtype: The type of room offered (e.g.~private room, shared room,
entire home/apt).

room\_shared: Whether the room is shared or not.

room\_private: Whether the room is private or not.

person\_capacity: The maximum number of people that can be accommodated
in a single listing.

host\_is\_superhost: Whether or not a particular host is identified as a
superhost on Airbnb.

multi: Whether multiple rooms are provided in one individual listing or
not.

biz: Whether a particular listing offers business facilities like
meeting area/conference rooms in addition to

cleanless\_rating: The rating associated with how clean an individual
property was after guests stayed at it.

guest\_satisfaction\_overall: The rating associated with how clean an
individual property was after guests stayed at it.

dist: The total quantity of bedrooms available among all properties
against a single hosting id.

metro\_dist: Distance from metro station associated with every rental
property.

attr\_index: attraction index

attr\_index\_norm: attraction index, normalized

rest\_index: restaurant index

rest\_index\_norm: restaurant index, normalized

lng Longitude measurement corresponding to each rental unit.

lat: Latitude measurement corresponding to each rental unit

\hypertarget{pre-processing-and-cleaning-the-data}{%
\section{Pre Processing and Cleaning the
Data}\label{pre-processing-and-cleaning-the-data}}

Before we could begin the analysis, we preprocessed the data to ensure
that it was consistent. This involved combining the 20 files to a single
table with a additional column of city\_day.

We also removed certain redundant columns such as room\_shared and
room\_private whose information is present in room\_type completely ans
exhaustively.

Next we have separated the outliers using IQR ranges.

We have also Dropped attr\_index and rest\_index because they were
already normalized and given as seperate attributes.

We have split the training and testing data here itself with 7:3 split
on constant seed.

\hypertarget{exploratory-data-analysis}{%
\section{Exploratory Data Analysis}\label{exploratory-data-analysis}}

We performed several analyses to identify the factors that affect Airbnb
pricing. Firstly, we used descriptive statistics to analyze the
distribution of the variables and examine any patterns or trends.

\hypertarget{outlier-analysis}{%
\subsection{Outlier Analysis}\label{outlier-analysis}}

\hypertarget{metro-dist-vs-real-sum}{%
\subsubsection{Metro Dist vs Real Sum}\label{metro-dist-vs-real-sum}}

We started the EDA with Outlier analysis, We have planned to analyse the
filtered data along with outlier data. Here outlier data represents the
hotel rooms with high prices.

In general the rooms that are closer to metro have comparatively higher
prices. But, in Rome city the distance to metro is almost same for both
categories of price.

\hypertarget{real-sum-vs-distance}{%
\subsubsection{Real Sum vs Distance}\label{real-sum-vs-distance}}

In general the pricey rooms are near to the centre of the city according
to the Scatterplot

\hypertarget{real-sum-vs-attraction-index-normal}{%
\subsubsection{Real Sum vs Attraction Index
Normal}\label{real-sum-vs-attraction-index-normal}}

The range of values falling b/w outliers and normal data is almost same
. So there isn't a relationship b/w attr\_index and realSum.

\hypertarget{real-sum-vs-restaurant-index-normal}{%
\subsubsection{Real Sum vs Restaurant Index
Normal}\label{real-sum-vs-restaurant-index-normal}}

There is no relationship between outliers and rest\_index

\hypertarget{room-type-vs-real-sum}{%
\subsubsection{Room Type Vs Real Sum}\label{room-type-vs-real-sum}}

The price of entire home/apt tend to be higher compared to other two
categories. And the count of entire home /apt is also more.

\hypertarget{room-type-vs-person-capacity}{%
\subsubsection{Room Type Vs Person
Capacity}\label{room-type-vs-person-capacity}}

The overall price is distributed similarly across the spectrum
irrespective of person\_capacity. But for some cities like london,
london\_weekdays, lisbon the price is higher with person capacity. So,
person capacity along with city will be an important variable for
determining price.

\hypertarget{real-sum-vs-host_is_superhost}{%
\subsubsection{Real Sum Vs
host\_is\_superhost}\label{real-sum-vs-host_is_superhost}}

The prices are spread across all the spectrum irrespective of
super\_host or not.

\hypertarget{real-sum-vs-multi}{%
\subsubsection{Real Sum Vs multi}\label{real-sum-vs-multi}}

The prices are similar irrespective of multi or not.

\hypertarget{real-sum-vs-biz}{%
\subsubsection{Real Sum vs biz}\label{real-sum-vs-biz}}

The prices are similar irrespective of biz or not.

\hypertarget{real-sum-vs-cleanliness}{%
\subsubsection{Real Sum vs Cleanliness}\label{real-sum-vs-cleanliness}}

The cleanliness rating doesn't really have an effect on price

\hypertarget{scatterplot-of-price-vs-guest-satisfaction-filtered-by-city}{%
\subsubsection{Scatterplot of Price vs Guest Satisfaction filtered by
city}\label{scatterplot-of-price-vs-guest-satisfaction-filtered-by-city}}

The plot depicts that there is no correlation of price with guest
satisfaction, good satisfaction rate is found across all the prices. In
some cities like lonon, we can see a group of reviews with low guest
satisfaction.

\hypertarget{real-sum-vs-bedroom-count}{%
\subsubsection{Real Sum Vs Bedroom
Count}\label{real-sum-vs-bedroom-count}}

No observable relation.

\hypertarget{non---outlier-analysis}{%
\subsection{Non - Outlier Analysis}\label{non---outlier-analysis}}

\hypertarget{boxplot-of-price-vs-city}{%
\subsubsection{Boxplot of Price Vs
City}\label{boxplot-of-price-vs-city}}

The highest prices in Europe are found in Amsterdam.

\hypertarget{density-plot-of-price-vs-room-type}{%
\subsubsection{Density plot of Price vs Room
type}\label{density-plot-of-price-vs-room-type}}

The prices of entire home are high comparatively

\hypertarget{scatterplot-of-prices-in-rome-w.r.t-latitude-and-longitude-during-weekdays}{%
\subsubsection{Scatterplot of Prices in Rome w.r.t Latitude and
Longitude during
weekdays}\label{scatterplot-of-prices-in-rome-w.r.t-latitude-and-longitude-during-weekdays}}

This plot is within expectations of general trends, which suggests
similar types of establishments (price and hospitality) tend be in
clusters.

\hypertarget{different-model-selection-and-training}{%
\section{Different Model Selection and
Training}\label{different-model-selection-and-training}}

We conducted a regression analysis to determine the variables that had
the most significant impact on Airbnb pricing.

Linear regression has given a R\^{}2 value of 0.21 which is really low.

The situation is same with Lasso step regression.

\hypertarget{conclusion-at-this-point-in-time}{%
\section{Conclusion at this Point in
Time}\label{conclusion-at-this-point-in-time}}

Even though EDA has given us good insights in price determinants, both
Linear and Lasso Step Regression are not good for this case which is to
be expected since all common data tends to be generally skewed Normal or
Guassian(to be tested).

Further modelling is required and will be conducted which includes
trying of different linear techniques and also models from different
family.

\end{document}
